\documentclass[11pt]{article} 
\usepackage[english]{babel}
\usepackage[utf8]{inputenc}
\usepackage[margin=0.5in]{geometry}
\usepackage{amsmath}
\usepackage{amsthm}
\usepackage{amsfonts}
\usepackage{amssymb}
\usepackage[usenames,dvipsnames]{xcolor}
\usepackage{graphicx}
\usepackage[siunitx]{circuitikz}
\usepackage{tikz}
\usepackage[colorinlistoftodos, color=orange!50]{todonotes}
\usepackage{hyperref}
\usepackage[numbers, square]{natbib}
\usepackage{fancybox}
\usepackage{epsfig}
\usepackage{soul}
\usepackage[framemethod=tikz]{mdframed}
\usepackage[shortlabels]{enumitem}
\usepackage[version=4]{mhchem}
\usepackage{multicol}

\usepackage{mathtools}
\usepackage{comment}
\usepackage{enumitem}
\usepackage[utf8]{inputenc}
\usepackage[linesnumbered,ruled,vlined]{algorithm2e}
\usepackage{listings}
\usepackage{color}
\usepackage[numbers]{natbib}
\usepackage{subfiles}
\usepackage{tkz-berge}


\newtheorem{prop}{Proposition}[section]
\newtheorem{thm}{Theorem}[section]
\newtheorem{lemma}{Lemma}[section]
\newtheorem{cor}{Corollary}[prop]

\theoremstyle{definition}
\newtheorem{definition}{Definition}

\theoremstyle{definition}
\newtheorem{required}{Problem}

\theoremstyle{definition}
\newtheorem{ex}{Example}


\setlength{\marginparwidth}{3.4cm}
%#########################################################

%To use symbols for footnotes
\renewcommand*{\thefootnote}{\fnsymbol{footnote}}
%To change footnotes back to numbers uncomment the following line
%\renewcommand*{\thefootnote}{\arabic{footnote}}

% Enable this command to adjust line spacing for inline math equations.
% \everymath{\displaystyle}

% _______ _____ _______ _      ______ 
%|__   __|_   _|__   __| |    |  ____|
%   | |    | |    | |  | |    | |__   
%   | |    | |    | |  | |    |  __|  
%   | |   _| |_   | |  | |____| |____ 
%   |_|  |_____|  |_|  |______|______|
%%%%%%%%%%%%%%%%%%%%%%%%%%%%%%%%%%%%%%%

\title{
\normalfont \normalsize 
\textsc{CSCI 3104 Fall 2021 \\ 
Instructor: Profs. Grochow and Waggoner} \\
[10pt] 
\rule{\linewidth}{0.5pt} \\[6pt] 
\huge Quiz- Standard 3 \\
\rule{\linewidth}{2pt}  \\[10pt]
}
%\author{Your Name}
\date{}

\begin{document}

\maketitle


%%%%%%%%%%%%%%%%%%%%%%%%%
%%%%%%%%%%%%%%%%%%%%%%%%%%
%%%%%%%%%%FILL IN YOUR NAME%%%%%%%
%%%%%%%%%%AND STUDENT ID%%%%%%%%
%%%%%%%%%%%%%%%%%%%%%%%%%%
\noindent
Due Date \dotfill TODO \\
Name \dotfill \textbf{John Blackburn} \\
Student ID \dotfill \textbf{jobl2177} \\


\tableofcontents

\section{Instructions}
 \begin{itemize}
	\item The solutions \textbf{should be typed}, using proper mathematical notation. We cannot accept hand-written solutions. \href{http://ece.uprm.edu/~caceros/latex/introduction.pdf}{Here's a short intro to \LaTeX.}
	\item You should submit your work through the \textbf{class Canvas page} only. Please submit one PDF file, compiled using this \LaTeX \ template.
	\item You may not need a full page for your solutions; pagebreaks are there to help Gradescope automatically find where each problem is. Even if you do not attempt every problem, please submit this document with no fewer pages than the blank template (or Gradescope has issues with it).

	\item You \textbf{may not collaborate with other students}. \textbf{Copying from any source is an Honor Code violation. Furthermore, all submissions must be in your own words and reflect your understanding of the material.} If there is any confusion about this policy, it is your responsibility to clarify before the due date. 

	\item Posting to \textbf{any} service including, but not limited to Chegg, Discord, Reddit, StackExchange, etc., for help on an assignment is a violation of the Honor Code.

	\item You \textbf{must} virtually sign the Honor Code (see Section \ref{HonorCode}). Failure to do so will result in your assignment not being graded.
\end{itemize}


\section{Honor Code (Make Sure to Virtually Sign)} \label{HonorCode}

\begin{required}
\noindent 
\begin{itemize}
\item My submission is in my own words and reflects my understanding of the material.
\item I have not collaborated with any other person.
\item I have not posted to external services including, but not limited to Chegg, Discord, Reddit, StackExchange, etc.
\item I have neither copied nor provided others solutions they can copy.
\end{itemize}

%\noindent In the specified region below, clearly indicate that you have upheld the Honor Code. Then type your name. 
\end{required}

\begin{proof}[Agreed (John Blackburn).]
%% Typing "I agree to the above," followed by your name is sufficient.
\end{proof}



\newpage
\section{Standard Dijkstra}

\subsection{Problem \ref{Dijkstra1}}
\begin{required} \label{Dijkstra1}
Consider the weighted graph $G(V, E, w)$ pictured below. Work through Dijkstra's algorithm on the following graph, using the source vertex $A$. 
\begin{itemize}
\item Clearly include the contents of the priority queue, as well as the distance from $A$ to each vertex at each iteration.
\item If you use a table to store the distances, clearly label the keys according to the vertex names rather than numeric indices (i.e., \texttt{dist[`B']} is more descriptive than \texttt{dist[`1']}).
\item You do \textbf{not} need to draw the graph at each iteration, though you are welcome to do so. [This may be helpful scratch work, which you do not need to include.]
\end{itemize}

\definecolor {processblue}{cmyk}{0.96,0,0,0}
\begin{center}
	\begin {tikzpicture}[-latex ,auto ,node distance =2 cm and 3cm ,on grid ,
	semithick ,
	state/.style ={ circle ,top color =white , bottom color = processblue!20 ,
	draw,processblue , text=blue , minimum width =1 cm}, scale=0.5]

	\node[state] (A) {$A$};
	\node[state] (B) [above right = of A] {$B$};
	\node[state] (C) [below right = of A] {$C$};
	\node[state] (G) [below right = of B] {$H$};
	\node[state] (D) [above right = of G] {$D$};
	\node[state] (E) [below right = of G] {$E$};
	\node[state] (F) [below right = of D] {$F$};
	
	\path (A) edge node[above] {$4$} (B);
	\path (A) edge node[above] {$8$} (C);
	\path (B) edge node[above] {$1$} (G);
	\path (B) edge node[above] {$6$} (D);
	\path (G) edge node[above] {$2$} (A);
	\path (E) edge node[above] {$1$} (C);
	\path (D) edge node[above] {$3$} (G);
	\path (D) edge node[above] {$10$} (F);
	\path (F) edge node[above] {$2$} (E);
	\path (G) edge node[above] {$5$} (E);
	\path (G) edge node[above] {$2$} (F);
	\path (G) edge node[above] {$2$} (C);
	
	\end{tikzpicture}  
\end{center}

\end{required}

\begin{proof}[Answer]
%Your answer here
Our source node is $A$. we begin by setting the distances for each node other than $A$ to $\infty$. We also set A as processed. 


Next we examine $A$'s neighbors, updating their found distances and placing them in the priority queue. The distance table is now:

$Q=[(B,4),(C,8)]$

We now have $B$ first in our queue with a distance of 4.

The found distances from A are now:

$(B,4),(C,8),(D,\infty),(E,\infty),(F,\infty),(H,\infty)$

Our next iteration brings us to vertex $B$, we mark it as processed, and begin to examine $B$'s unprocessed neighbors. Them being $H$ and $D$. Now we update our queue to add these new neighbors. $H$ will go first because it has the lowest distance from $A$.

$Q=[(H,5), (C,8), (D,10)]$

This brings our distances to:

$(B,4),(C,8),(D,10),(E,\infty),(F,\infty),(H,5)$


Next we remove $H$ from the queue mark it as processed and examine the unprocessed neighbors. We examine $C$ and find that we can decrease the distance to $7$ due to a new shorter path being found. Then we examine $F$ and mark it's distance as $7$. And finally we examine $E$, and mark it's distance to 10. We then rework the queue to hold the new vertex's. Our new queue is:

$Q=[(F,7), (C,7), (D,10), (E,10)]$

Our new distances are now:

$(B,4),(C,7),(D,10),(E,10),(F,7),(H,5)$

Our next iteration brings us to vertex $F$. we process $F$. We visit all the unprocessed neighbors of $F$, them being only $E$. We update the distance of $E$ to be $9$ because a new path has less distance than the previously found. Our new queue is:

$Q=[(C,7), (E,9), (D,10)]$


Distances:

$(B,4),(C,7),(D,10),(E,9),(F,7),(H,5)$

Our next iteration is on vertex $C$. We process $C$. We check all unprocessed neighbors, and we see that $C$ has no neighbors. SO we move to the next iteration. Queue:

$Q=[(E,9), (D,10)]$


Distances:

$(B,4),(C,7),(D,10),(E,9),(F,7),(H,5)$


Our next iteration is on vertex $E$. we mark it as processed. We check all unprocessed neighbors, and find we don't have any. We move on. Queue:

$Q=[(D,10)]$

Distances:

$(B,4),(C,7),(D,10),(E,9),(F,7),(H,5)$

Our next iteration is on vertex $D$. we mark it as processed. check unprocessed neighbors, we find all it's neighbors have been processed. We now longer have any unprocessed vertices and nothing left in our queue. Therefore we have found all minimum distances to each node from vertex$A$. Final Distances:

$(B,4),(C,7),(D,10),(E,9),(F,7),(H,5)$
\end{proof}




%%%%%%%%%%%%%%%%%%%%%%%%%%%%%%%%%%%%%%%%%%%%%%%%%%
\end{document} % NOTHING AFTER THIS LINE IS PART OF THE DOCUMENT



