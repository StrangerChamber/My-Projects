\documentclass[11pt]{article} 
\usepackage[english]{babel}
\usepackage[utf8]{inputenc}
\usepackage[margin=0.5in]{geometry}
\usepackage{amsmath}
\usepackage{amsthm}
\usepackage{amsfonts}
\usepackage{amssymb}
\usepackage[usenames,dvipsnames]{xcolor}
\usepackage{graphicx}
\usepackage[siunitx]{circuitikz}
\usepackage{tikz}
\usetikzlibrary{calc,arrows.meta}
\usepackage[colorinlistoftodos, color=orange!50]{todonotes}
\usepackage{hyperref}
\usepackage[numbers, square]{natbib}
\usepackage{fancybox}
\usepackage{epsfig}
\usepackage{soul}
\usepackage[framemethod=tikz]{mdframed}
\usepackage[shortlabels]{enumitem}
\usepackage[version=4]{mhchem}
\usepackage{multicol}
\usepackage{mathtools}
\usepackage{comment}
\usepackage{enumitem}
\usepackage[utf8]{inputenc}
\usepackage{listings}
\usepackage{color}
\usepackage[numbers]{natbib}
\usepackage{subfiles}
\usepackage{tkz-berge}
\usepackage{algorithm}
\usepackage[noend]{algpseudocode}


\newtheorem{prop}{Proposition}[section]
\newtheorem{thm}{Theorem}[section]
\newtheorem{lemma}{Lemma}[section]
\newtheorem{cor}{Corollary}[prop]

\theoremstyle{definition}
\newtheorem{definition}{Definition}

\theoremstyle{definition}
\newtheorem{required}{Problem}

\theoremstyle{definition}
\newtheorem{ex}{Example}

\tikzset{
	vertex/.style={circle,draw,minimum size=16, inner sep=0pt,font=\normalsize},
	every node/.style={draw=none,rectangle,font=\scriptsize,outer sep=0pt,inner sep=2pt},
	directed/.style={arrows={-Stealth[length=7pt]},font=\small},
	caption/.style={text width=6cm,align=center,rectangle,draw}
}


\setlength{\marginparwidth}{3.4cm}
%#########################################################

%To use symbols for footnotes
\renewcommand*{\thefootnote}{\fnsymbol{footnote}}
%To change footnotes back to numbers uncomment the following line
%\renewcommand*{\thefootnote}{\arabic{footnote}}

% Enable this command to adjust line spacing for inline math equations.
% \everymath{\displaystyle}

% _______ _____ _______ _      ______ 
%|__   __|_   _|__   __| |    |  ____|
%   | |    | |    | |  | |    | |__   
%   | |    | |    | |  | |    |  __|  
%   | |   _| |_   | |  | |____| |____ 
%   |_|  |_____|  |_|  |______|______|
%%%%%%%%%%%%%%%%%%%%%%%%%%%%%%%%%%%%%%%

\title{
\normalfont \normalsize 
\textsc{CSCI 3104 Fall 2021 \\ 
Instructors: Profs. Grochow and Waggoner} \\
[10pt] 
\rule{\linewidth}{0.5pt} \\[6pt] 
\huge Final- Standard 30 \\
\rule{\linewidth}{2pt}  \\[10pt]
}
%\author{Your Name}
\date{}

\begin{document}
\definecolor {processblue}{cmyk}{0.96,0,0,0}
\definecolor{processred}{rgb}{200, 0, 0}
\definecolor{processgreen}{rgb}{0, 255, 0}
\DeclareGraphicsExtensions{.png}
\DeclareGraphicsExtensions{.gif}
\DeclareGraphicsExtensions{.jpg}

\maketitle


%%%%%%%%%%%%%%%%%%%%%%%%%
%%%%%%%%%%%%%%%%%%%%%%%%%%
%%%%%%%%%%FILL IN YOUR NAME%%%%%%%
%%%%%%%%%%AND STUDENT ID%%%%%%%%
%%%%%%%%%%%%%%%%%%%%%%%%%%
\noindent
Due Date \dotfill TODO \\
Name \dotfill \textbf{John Blackburn} \\
Student ID \dotfill \textbf{Jobl2177} \\


\tableofcontents

\section{Instructions}
 \begin{itemize}
	\item The solutions \textbf{should be typed}, using proper mathematical notation. We cannot accept hand-written solutions. \href{http://ece.uprm.edu/~caceros/latex/introduction.pdf}{Here's a short intro to \LaTeX.}
	\item You should submit your work through the \textbf{class Canvas page} only. Please submit one PDF file, compiled using this \LaTeX \ template.
	\item You may not need a full page for your solutions; pagebreaks are there to help Gradescope automatically find where each problem is. Even if you do not attempt every problem, please submit this document with no fewer pages than the blank template (or Gradescope has issues with it).

	\item You \textbf{may not collaborate with other students}. \textbf{Copying from any source is an Honor Code violation. Furthermore, all submissions must be in your own words and reflect your understanding of the material.} If there is any confusion about this policy, it is your responsibility to clarify before the due date. 

	\item Posting to \textbf{any} service including, but not limited to Chegg, Discord, Reddit, StackExchange, etc., for help on an assignment is a violation of the Honor Code.

	\item You \textbf{must} virtually sign the Honor Code (see Section \ref{HonorCode}). Failure to do so will result in your assignment not being graded.
\end{itemize}


\section{Honor Code (Make Sure to Virtually Sign)} \label{HonorCode}

\begin{required}
\begin{itemize}
\item My submission is in my own words and reflects my understanding of the material.
\item Any collaborations and external sources have been clearly cited in this document.
\item I have not posted to external services including, but not limited to Chegg, Reddit, StackExchange, etc.
\item I have neither copied nor provided others solutions they can copy.
\end{itemize}

%\noindent In the specified region below, clearly indicate that you have upheld the Honor Code. Then type your name. 
\end{required}

\begin{proof}[Agreed John Blackburn.]
%% Typing "I agree to the above," followed by your name is sufficient.
\end{proof}


\newpage
\section{Standard 30- Computational Complexity: Structure}
\begin{required}
\noindent \\
\begin{enumerate}[label=(\alph*)]
    \item Suppose that $\mathcal{C}$ is a complexity class such that $\textsf{P} \subseteq \mathcal{C}$ (so every problem in $\textsf{P}$ is contained in $\mathcal{C}$). Suppose that there is a problem $L$ that is $\mathcal{C}$-complete under polynomial-time reductions. Recall that this means: 
\begin{itemize}
\item $L \in \mathcal{C}$; and 
\item For all problems $K \in \mathcal{C}$, $K$ can be reduced to $L$ in polynomial time. 
\end{itemize}

Show that if $L \in \textsf{P}$, then $\mathcal{C} \subseteq \textsf{P}$. (Since we are assuming that $\textsf{P} \subseteq \mathcal{C}$ this means that $\textsf{P} = \mathcal{C}$.)

\begin{proof}
%Your answer here
We know $L \in \textsf{P}$ and that all problems $K \in \mathcal{C}$ can be reduced to L. Thus, we know that $L \in \mathcal{C}$ as well. We also know that we can reduce $L$ to any problem in $\mathcal{C}$. Thus making any problem in C also in P. From there we can determine that $\mathcal{C} \subseteq \textsf{P}$
\end{proof}
    

\vskip 15pt
    \item We say that a language $L \in \textsf{EXPTIME}$ if there is an algorithm $A$ that (i) correctly decides whether a given input string $x$ belongs to $L$, and (ii) $A$ runs in time at most $2^{n^{c}}$ where $c$ is a constant (depends only on $L$) and $n := |x|$ is the size of the input in bits. It is known that $\textsf{P} \subseteq \textsf{EXPTIME}$ but also that $\textsf{P} \ne \textsf{EXPTIME}$. Use these facts and part (a) to argue that no $\textsf{EXPTIME}$-complete language belongs to $\textsf{P}$.

\begin{proof}
%Your answer here
Any EXPTIME-complete language is a problem that any problem in EXPTIME can be reduced to. Thus, if all problems in P can be reduced to an EXPTIME-complete problem we would be able to prove that P = EXPTIME. Therefore we can see that no EXPTIME-complete problem can belong to P because that would mean P = EXPTIME. 
\end{proof}


\end{enumerate}
\end{required}


%Include an Image: \includegraphics{ImageFileName}
%Include an Image and Rotate 90 degree: \includegraphics[angle=90]{ImageFileName}
%Include an Image, Rotate by 180 degrees, and scale by 50\% \includegraphics[scale=0.5, angle=90]{ImageFileName}



%%%%%%%%%%%%%%%%%%%%%%%%%%%%%%%%%%%%%%%%%%%%%%%%%%
\end{document} % NOTHING AFTER THIS LINE IS PART OF THE DOCUMENT



