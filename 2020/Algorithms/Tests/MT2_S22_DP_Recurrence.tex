\documentclass[11pt]{article} 
\usepackage[english]{babel}
\usepackage[utf8]{inputenc}
\usepackage[margin=0.5in]{geometry}
\usepackage{amsmath}
\usepackage{amsthm}
\usepackage{amsfonts}
\usepackage{amssymb}
\usepackage[usenames,dvipsnames]{xcolor}
\usepackage{graphicx}
\usepackage[siunitx]{circuitikz}
\usepackage{tikz}
\usetikzlibrary{calc,arrows.meta}
\usepackage[colorinlistoftodos, color=orange!50]{todonotes}
\usepackage{hyperref}
\usepackage[numbers, square]{natbib}
\usepackage{fancybox}
\usepackage{epsfig}
\usepackage{soul}
\usepackage[framemethod=tikz]{mdframed}
\usepackage[shortlabels]{enumitem}
\usepackage[version=4]{mhchem}
\usepackage{multicol}
\usepackage{mathtools}
\usepackage{comment}
\usepackage{enumitem}
\usepackage[utf8]{inputenc}
\usepackage{listings}
\usepackage{color}
\usepackage[numbers]{natbib}
\usepackage{subfiles}
\usepackage{tkz-berge}
\usepackage{algorithm}
\usepackage[noend]{algpseudocode}


\newtheorem{prop}{Proposition}[section]
\newtheorem{thm}{Theorem}[section]
\newtheorem{lemma}{Lemma}[section]
\newtheorem{cor}{Corollary}[prop]

\theoremstyle{definition}
\newtheorem{definition}{Definition}

\theoremstyle{definition}
\newtheorem{required}{Problem}

\theoremstyle{definition}
\newtheorem{ex}{Example}

\tikzset{
	vertex/.style={circle,draw,minimum size=16, inner sep=0pt,font=\normalsize},
	every node/.style={draw=none,rectangle,font=\scriptsize,outer sep=0pt,inner sep=2pt},
	directed/.style={arrows={-Stealth[length=7pt]},font=\small},
	caption/.style={text width=6cm,align=center,rectangle,draw}
}


\setlength{\marginparwidth}{3.4cm}
%#########################################################

%To use symbols for footnotes
\renewcommand*{\thefootnote}{\fnsymbol{footnote}}
%To change footnotes back to numbers uncomment the following line
%\renewcommand*{\thefootnote}{\arabic{footnote}}

% Enable this command to adjust line spacing for inline math equations.
% \everymath{\displaystyle}

% _______ _____ _______ _      ______ 
%|__   __|_   _|__   __| |    |  ____|
%   | |    | |    | |  | |    | |__   
%   | |    | |    | |  | |    |  __|  
%   | |   _| |_   | |  | |____| |____ 
%   |_|  |_____|  |_|  |______|______|
%%%%%%%%%%%%%%%%%%%%%%%%%%%%%%%%%%%%%%%

\title{
\normalfont \normalsize 
\textsc{CSCI 3104 Fall 2021 \\ 
Instructors: Profs. Grochow and Waggoner} \\
[10pt] 
\rule{\linewidth}{0.5pt} \\[6pt] 
\huge Midterm 2- Standard 22 \\
\rule{\linewidth}{2pt}  \\[10pt]
}
%\author{Your Name}
\date{}

\begin{document}
\definecolor {processblue}{cmyk}{0.96,0,0,0}
\definecolor{processred}{rgb}{200, 0, 0}
\definecolor{processgreen}{rgb}{0, 255, 0}
\DeclareGraphicsExtensions{.png}
\DeclareGraphicsExtensions{.gif}
\DeclareGraphicsExtensions{.jpg}

\maketitle


%%%%%%%%%%%%%%%%%%%%%%%%%
%%%%%%%%%%%%%%%%%%%%%%%%%%
%%%%%%%%%%FILL IN YOUR NAME%%%%%%%
%%%%%%%%%%AND STUDENT ID%%%%%%%%
%%%%%%%%%%%%%%%%%%%%%%%%%%
\noindent
Due Date \dotfill TODO \\
Name \dotfill \textbf{john blackburn} \\
Student ID \dotfill \textbf{jobl2177} \\


\tableofcontents

\section{Instructions}
 \begin{itemize}
	\item The solutions \textbf{should be typed}, using proper mathematical notation. We cannot accept hand-written solutions. \href{http://ece.uprm.edu/~caceros/latex/introduction.pdf}{Here's a short intro to \LaTeX.}
	\item You should submit your work through the \textbf{class Canvas page} only. Please submit one PDF file, compiled using this \LaTeX \ template.
	\item You may not need a full page for your solutions; pagebreaks are there to help Gradescope automatically find where each problem is. Even if you do not attempt every problem, please submit this document with no fewer pages than the blank template (or Gradescope has issues with it).

	\item You \textbf{may not collaborate with other students}. \textbf{Copying from any source is an Honor Code violation. Furthermore, all submissions must be in your own words and reflect your understanding of the material.} If there is any confusion about this policy, it is your responsibility to clarify before the due date. 

	\item Posting to \textbf{any} service including, but not limited to Chegg, Discord, Reddit, StackExchange, etc., for help on an assignment is a violation of the Honor Code.

	\item You \textbf{must} virtually sign the Honor Code (see Section \ref{HonorCode}). Failure to do so will result in your assignment not being graded.
\end{itemize}


\section{Honor Code (Make Sure to Virtually Sign)} \label{HonorCode}

\begin{required}
\noindent 
\begin{itemize}
\item My submission is in my own words and reflects my understanding of the material.
\item I have not collaborated with any other person.
\item I have not posted to external services including, but not limited to Chegg, Discord, Reddit, StackExchange, etc.
\item I have neither copied nor provided others solutions they can copy.
\end{itemize}

%\noindent In the specified region below, clearly indicate that you have upheld the Honor Code. Then type your name. 
\end{required}

\begin{proof}[Agreed (john blackburn).]
%% Typing "I agree to the above," followed by your name is sufficient.
\end{proof}


\newpage
\section{Standard 22- DP: Write Recurrence}

\begin{required}
A \textit{transmission} on $n$ seconds (where $n \geq 0$ is an integer) is an ordered sequence of signals, drawn from $s_{1}$ and $s_{2}$, where $s_{1}$ takes one second to send and $s_{2}$ takes two seconds to send. Assume there is no delay between signals. Construct a recurrence to count the number of distinct transmissions we can send in $n$ seconds. Make sure to include your base cases. Justify your recurrence.
\end{required}

\begin{proof}[Answer]
%Your answer here
Let $T(n)$ denote the recurrence.

when $n=0$ the only transmission we can send is no transmission so $T(0) = 1$. When $n=1$ $T(1) = 2$ because we can send no message and a message using only $s_1$. As for when $n>1$ we have two cases that it breaks down to:

Case 1: $s_1$ is our first signal sent, the rest of our sequence is now of length $n-1$, therefore it is the same problem as $T(n-1)$ with $s_1$ in front if it. And there are $T(n-1)$ ways to choose this. 

Case 2: $s_2$ is the first signal sent, the rest of our sequence is now of length $n-2$, therefore it is now the same problem as $T(n-2)$ with $s_2$ in front of it. And there are $T(n-2)$ ways to select this.

Therefore the entire recurrence is:

\begin{align*}
T_{n} &= \begin{cases}
\text{1} & : \text{n=0}, \\
\text{2} & : \text{n=1}, \\
\text{T(n-1)+T(n-2)} & : \text{$n>1$}. 
\end{cases}
\end{align*}
\end{proof}


%LaTeX code to render a recurrence
\begin{comment}
\begin{align*}
f_{n} &= \begin{cases}
\text{case1} & : \text{condition1}, \\
\text{case2} & : \text{condition2}. 
\end{cases}
\end{align*}
\end{comment}

%%%%%%%%%%%%%%%%%%%%%%%%%%%%%%%%%%%%%%%%%%%%%%%%%%
\end{document} % NOTHING AFTER THIS LINE IS PART OF THE DOCUMENT



