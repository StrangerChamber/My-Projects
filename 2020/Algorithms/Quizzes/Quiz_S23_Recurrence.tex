\documentclass[11pt]{article} 
\usepackage[english]{babel}
\usepackage[utf8]{inputenc}
\usepackage[margin=0.5in]{geometry}
\usepackage{amsmath}
\usepackage{amsthm}
\usepackage{amsfonts}
\usepackage{amssymb}
\usepackage[usenames,dvipsnames]{xcolor}
\usepackage{graphicx}
\usepackage[siunitx]{circuitikz}
\usepackage{tikz}
\usepackage[colorinlistoftodos, color=orange!50]{todonotes}
\usepackage{hyperref}
\usepackage[numbers, square]{natbib}
\usepackage{fancybox}
\usepackage{epsfig}
\usepackage{soul}
\usepackage[framemethod=tikz]{mdframed}
\usepackage[shortlabels]{enumitem}
\usepackage[version=4]{mhchem}
\usepackage{multicol}

\usepackage{mathtools}
\usepackage{comment}
\usepackage{enumitem}
\usepackage[utf8]{inputenc}
\usepackage[linesnumbered,ruled,vlined]{algorithm2e}
\usepackage{listings}
\usepackage{color}
\usepackage[numbers]{natbib}
\usepackage{subfiles}
\usepackage{tkz-berge}


\newtheorem{prop}{Proposition}[section]
\newtheorem{thm}{Theorem}[section]
\newtheorem{lemma}{Lemma}[section]
\newtheorem{cor}{Corollary}[prop]

\theoremstyle{definition}
\newtheorem{definition}{Definition}

\theoremstyle{definition}
\newtheorem{required}{Problem}

\theoremstyle{definition}
\newtheorem{ex}{Example}


\setlength{\marginparwidth}{3.4cm}
%#########################################################

%To use symbols for footnotes
\renewcommand*{\thefootnote}{\fnsymbol{footnote}}
%To change footnotes back to numbers uncomment the following line
%\renewcommand*{\thefootnote}{\arabic{footnote}}

% Enable this command to adjust line spacing for inline math equations.
% \everymath{\displaystyle}

% _______ _____ _______ _      ______ 
%|__   __|_   _|__   __| |    |  ____|
%   | |    | |    | |  | |    | |__   
%   | |    | |    | |  | |    |  __|  
%   | |   _| |_   | |  | |____| |____ 
%   |_|  |_____|  |_|  |______|______|
%%%%%%%%%%%%%%%%%%%%%%%%%%%%%%%%%%%%%%%

\title{
\normalfont \normalsize 
\textsc{CSCI 3104 Fall 2021 \\ 
Instructors: Profs. Grochow and Waggoner} \\
[10pt] 
\rule{\linewidth}{0.5pt} \\[6pt] 
\huge Quiz- Standard 23 \\
\rule{\linewidth}{2pt}  \\[10pt]
}
%\author{Your Name}
\date{}

\begin{document}
\definecolor {processblue}{cmyk}{0.96,0,0,0}
\definecolor{processred}{rgb}{200, 0, 0}
\definecolor{processgreen}{rgb}{0, 255, 0}
\DeclareGraphicsExtensions{.png}
\DeclareGraphicsExtensions{.gif}
\DeclareGraphicsExtensions{.jpg}

\maketitle


%%%%%%%%%%%%%%%%%%%%%%%%%
%%%%%%%%%%%%%%%%%%%%%%%%%%
%%%%%%%%%%FILL IN YOUR NAME%%%%%%%
%%%%%%%%%%AND STUDENT ID%%%%%%%%
%%%%%%%%%%%%%%%%%%%%%%%%%%
\noindent
Due Date \dotfill TODO \\
Name \dotfill \textbf{john blackburn} \\
Student ID \dotfill \textbf{jobl2177} \\


\tableofcontents

\section{Instructions}
 \begin{itemize}
	\item The solutions \textbf{should be typed}, using proper mathematical notation. We cannot accept hand-written solutions. \href{http://ece.uprm.edu/~caceros/latex/introduction.pdf}{Here's a short intro to \LaTeX.}
	\item You should submit your work through the \textbf{class Canvas page} only. Please submit one PDF file, compiled using this \LaTeX \ template.
	\item You may not need a full page for your solutions; pagebreaks are there to help Gradescope automatically find where each problem is. Even if you do not attempt every problem, please submit this document with no fewer pages than the blank template (or Gradescope has issues with it).

	\item You \textbf{may not collaborate with other students}. \textbf{Copying from any source is an Honor Code violation. Furthermore, all submissions must be in your own words and reflect your understanding of the material.} If there is any confusion about this policy, it is your responsibility to clarify before the due date. 

	\item Posting to \textbf{any} service including, but not limited to Chegg, Discord, Reddit, StackExchange, etc., for help on an assignment is a violation of the Honor Code.

	\item You \textbf{must} virtually sign the Honor Code (see Section \ref{HonorCode}). Failure to do so will result in your assignment not being graded.
\end{itemize}

\tikzset{>={Stealth[length=7pt]}}
\tikzset{
    vertex/.style={circle,draw,minimum size=16,inner sep=0pt,font=\normalsize},
    edgelabel/.style={rectangle,draw=none,font=\footnotesize,outer sep=0pt},
    every node/.style={vertex},
    every edge quotes/.append style={edgelabel},
    every to/.append style={every node/.style={edgelabel}},
    wide/.style={line width=4pt,>={Stealth[length=18pt]}},
    directed/.style={arrows={->},font=\small},
    }

\section{Honor Code (Make Sure to Virtually Sign)} \label{HonorCode}

\begin{required}
\noindent 
\begin{itemize}
\item My submission is in my own words and reflects my understanding of the material.
\item I have not collaborated with any other person.
\item I have not posted to external services including, but not limited to Chegg, Discord, Reddit, StackExchange, etc.
\item I have neither copied nor provided others solutions they can copy.
\end{itemize}

%\noindent In the specified region below, clearly indicate that you have upheld the Honor Code. Then type your name. 
\end{required}

\begin{proof}[Agreed (john blackburn).]
%% Typing "I agree to the above," followed by your name is sufficient.
\end{proof}


\newpage
\section{Standard 23- DP: Use Recurrence to Solve}

\begin{required} \label{Problem2}
Recall the \textsf{Rod Cutting Problem} from class. 
\begin{itemize}
\item \textsf{Instance:} Let $n \geq 0$ be an integer where $n$ is the length of the rod, and let $p_{1}, p_{2}, \ldots, p_{n}$ be non-negative real numbers. Here, $p_{i}$ is the price of selling a rod of length $i$.
\item \textsf{Solution:} The maximum revenue, which we denote $r_{n}$, obtained by cutting the rod into pieces of integer lengths and selling the pieces.
\end{itemize}

\noindent Now suppose that our cutting tool is \textbf{malfunctioning} and \textbf{we are limited in the cuts we can make}.  We can always cut off a piece of length $1$, and additionally we can cut rods into halves (even $n$) or nearly halves (odd $n$): $\lfloor\frac{n}{2}\rfloor$ and $\lceil\frac{n}{2}\rceil$.  A recurrence relation describing $r_n$ is:

\[
r_{n} =
\begin{cases} 
	0 & \text{for } n=0,\\
	\max(p_n, r_1 + r_{n-1}, 2r_{n/2}) & \text{for even } n > 0, \\
        \max(p_n, r_1 + r_{n-1}, r_{(n-1)/2} + r_{(n+1)/2}) & \text{for odd } n > 0.
\end{cases}
\]

Suppose we have a rod of length $6$, with prices given as follows:
\[ p_{1} = 1, \quad p_{2} = 6, \quad p_{3} = 8, \quad p_{4} = 10, \quad p_{5} = 12, \quad p_{6} = 15. \] 
Using a bottom-up dynamic programming approach, determine the maximum revenue $r_{6}$ obtained by cutting up this rod \textbf{under the limited cut assumption described above}. For your convenience, we have provided the lookup table for you to use. You may also hand-draw your lookup tables. 

\noindent Clearly show all work for how you filled in each cell of the lookup table.

\end{required}

\begin{proof}[Answer]
\begin{center}
	\begin{tabular}{|c|c|c|c|c|c|c|}
	\hline
	$r_0$ & $r_{1}$ & $r_{2}$ & $r_{3}$ & $r_{4}$ & $r_{5}$ & $r_{6}$ \\ \hline
        0&1 & 6& 8&12 &14 &16 \\ \hline
	\end{tabular}
\end{center}

To fill in the lookup table I started from the base case $r_0$ which is equivalent to 0 which is defined above in the recurrence relation. The next case was $r_1$ which is equivalent to $\max(p_1, r_1 + r_{0}, r_{0} + r_{1})$ which comes out to $\max(1, 1 + 0, 0 + 1) = 1 = r_1$. we can now move to $r_2$, $r_2 = \max(p_n, r_1 + r_{n-1}, 2r_{n/2}) = \max(p_2, r_1 + r_{1}, 2r_{1}) = \max(6, 2, 2) = 6$.

Now for $r_3$: $r_3 =  \max(p_n, r_1 + r_{n-1}, r_{(n-1)/2} + r_{(n+1)/2}) =  \max(p_3, r_1 + r_{2}, r_{1} + r_{2}) = max(8, 7, 7) = 8$.

$r_4 = \max(p_n, r_1 + r_{n-1}, 2r_{n/2}) = \max(p_4, r_1 + r_{3}, 2r_{2}) = \max(10, 9, 12) = 12$

$ r_5 =  \max(p_n, r_1 + r_{n-1}, r_{(n-1)/2} + r_{(n+1)/2}) =  \max(p_5, r_1 + r_{4}, r_{2} + r_{3}) = max(12, 13, 14) = 14$

$r_6 = \max(p_n, r_1 + r_{n-1}, 2r_{n/2}) = \max(p_6, r_1 + r_{5}, 2r_{3}) = \max(15, 15, 16) = 16$

Thus completes all entries for the lookup table solved in a bottom up fashion. So, the maximum revenue $r_6$ can aquire is 16 units.

\end{proof}

%%%%%%%%%%%%%%%%%%%%%%%%%%%%%%%%%%%%%%%%%%%%%%%%%%
\end{document} % NOTHING AFTER THIS LINE IS PART OF THE DOCUMENT



